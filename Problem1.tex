\documentclass[a4paper,10pt]{article}
\usepackage[utf8]{inputenc}
\usepackage{geometry}
\usepackage[canada]{babel}
\usepackage[T1]{fontenc}
 \geometry{
 a4paper,
 total={170mm,257mm},
 left=15mm,
 top=15mm,
 right=15mm,
 bottom=15mm
 }

\title{SOEN 6011 Project - Calculator}
\author{Suruthi Raju - 40084709}

\date{}

\begin{document}

\maketitle

{\fontsize{12}{16}\selectfont This document is shows basic understanding on the functions that is used in Calculator: \\} 

{\Large\textbf{Function:}\\}
\newline
\indent\indent
{\fontsize{12}{16}\selectfont Given functionality for calculation is  a\textsuperscript{b \textsuperscript{x}}\\}

{\Large\textbf{Definition:}\\}
\newline
\indent\indent
{\fontsize{12}{16}\selectfont As per the given function, a and b are real constants and x is a real variable. Real Constants are real number which can be positive, negative or zero, which are fixed values. Whereas Real variables are not fixed values, which may change due to environment. In this project we are calculating the a power of b and b power x. \\}
\newline
\indent\indent
{\fontsize{12}{16}\selectfont b \textsuperscript{x} = b*....*b which is x times \\}
\newline
\indent\indent
{\fontsize{12}{16}\selectfont lets says the above value is c = b \textsuperscript{x} \\}
\newline
\indent\indent
{\fontsize{12}{16}\selectfont a \textsuperscript{c} = a*....*a which is c times \\}

{\Large\textbf{Domain and Co-Domain of function:} \\}
\newline
\indent\indent
{\fontsize{12}{16}\selectfont Domain of the power function depend upon value of the power x and b\textsuperscript{x} \\}
\newline
\indent\indent
{\fontsize{12}{16}\selectfont Case 1: p is a non-negative integer = then domain is all real numbers  \\ }
\newline
\indent\indent
{\fontsize{12}{16}\selectfont Case 2: p is a negative integer = then  domain is all real numbers not including zero  \\}


\indent\indent
{\fontsize{12}{16}\selectfont Case 3: p is a rational number in lowest terms as r/s and s is even = then when p>0 domain is non-negative real numbers, when p<0 domain is positive real numbers  \\}


\indent\indent
{\fontsize{12}{16}\selectfont Case 4: p is a rational number in lowest terms as r/s and s is odd = then when p>0 domain is all real numbers, when p<0 domain is all real numbers not including zero  \\}


\indent\indent
{\fontsize{12}{16}\selectfont Case 5: p is an  irrational number = then when p>0 domain is all non-negative real numbers, when p<0 domain is all positive real numbers  \\}

{\Large\textbf{Characteristics:}\\}
\newline
\indent\indent
{\fontsize{12}{16}\selectfont The properties is known as Exponents power rule \\}
\newline
\indent\indent
{\fontsize{12}{16}\selectfont For example: a\textsuperscript{b \textsuperscript{x}} is also can be written as a\textsuperscript{(b \textsuperscript{x})}}

\indent\indent\indent
{\fontsize{12}{16}\selectfont Since a and b are constant lets say 3 and 2 respectively \\}
\indent\indent\indent
{\fontsize{12}{16}\selectfont Where as x are variable it can vary lets say 4  \\}
\newline
\indent\indent\indent
{\fontsize{12}{16}\selectfont So the above example can be written as 3\textsuperscript{2 \textsuperscript{4}} or 3\textsuperscript{(2 \textsuperscript{4})} \\ }

\indent\indent\indent
{\fontsize{12}{16}\selectfont so the result is 3\textsuperscript{(2*2*2*2)} that is equal to 3\textsuperscript{16} which is 3*....*3 => 16 times }
\pagebreak

\newpage
{\Large\textbf{Requirements:}\\}
\newline
\indent\indent
{\fontsize{12}{16}\selectfont Requirement 1: Constant a and b are non-negative integer \\}
\newline
\indent\indent
{\fontsize{12}{16}\selectfont Requirement 2: Constant a is a negative integer. Constant b is a negative integer, when x!=0; \\}
\newline
\indent\indent
{\fontsize{12}{16}\selectfont Requirement 3: a is a rational number in lowest terms as b=r/s or x=r/s and s is even = then when a>0 domain is non-negative real numbers, when a<0 domain is positive real numbers\\}
\newline
\indent\indent
{\fontsize{12}{16}\selectfont Requirement 4: a is a rational number in lowest terms as b=r/s or x=r/s and s is odd = then when a>0 domain is all real numbers, when a<0 domain is all real numbers not including zero \\}
\newline
\indent\indent
{\fontsize{12}{16}\selectfont Requirement 5: a is an  irrational number = then when a>0 domain is all non-negative real numbers, when a<0 domain is all positive real numbers \\}

{\Large\textbf{Assumptions:}\\}
\newline
\indent\indent
{\fontsize{12}{16}\selectfont Assumption 1: The variable value x should not be negative when constant values are zero \\}
\newline
\indent\indent
{\fontsize{12}{16}\selectfont Assumption 2: The variable value x should not be zero \\}
\newline
\indent\indent
{\fontsize{12}{16}\selectfont Assumption 3: The constant value a and b should not be zero \\}

\end{document}
